% !Mode::"TeX:UTF-8"
\documentclass[onecolumn,landscape,UTF8]{ctexart}
\usepackage{lastpage}
%\usepackage{times} %use the Times New Roman fonts
\usepackage{color}
%\usepackage{placeins}
\usepackage{ulem}
\usepackage{titlesec}
\usepackage{graphicx}
\usepackage{colortbl}
\usepackage{listings}
\usepackage{makecell}
\usepackage{indentfirst}
\usepackage{fancyhdr}
\usepackage{setspace} % 行间距
\usepackage{bm}%\boldsymbol 粗体
% 数学
\usepackage{amsmath,amsfonts,amsmath,amssymb,times}
\usepackage{txfonts}
\usepackage{enumerate}% 编号
\usepackage{tikz,pgfplots} %绘图
\usepackage{tkz-euclide,pgfplots}
\usetikzlibrary{automata,positioning}
\usepackage[paperwidth=18.4cm,paperheight=26cm,top=1.5cm,bottom=2cm,right=2cm]{geometry} % 单页
%\usepackage[paperwidth=36.8cm,paperheight=26cm,top=2.5cm,bottom=2cm,right=2cm]{geometry}
%\lstset{language=C,keywordstyle=\color{red},showstringspaces=false,rulesepcolor=\color{green}}
%\oddsidemargin=0.5cm   %奇数页页边距
%\evensidemargin=0.5cm %偶数页页边距
%%\textwidth=14.5cm        %文本的宽度 单页
%\textwidth=30cm        %文本的宽度 单页

\newsavebox{\zdx}%装订线

\newcommand{\putzdx}{\marginpar{
		\parbox{1cm}{\vspace{-1.6cm}
			\rotatebox[origin=c]{90}{
				\usebox{\zdx}
		}}
}}

\newcommand{\blank}{\uline{\textcolor{white}{a}\ \textcolor{white}{a}\ \textcolor{white}{a}\ \textcolor{white}{a}\ \textcolor{white}{a}\ \textcolor{white}{a}\ \textcolor{white}{a}\ \textcolor{white}{a}\ \textcolor{white}{a}\ \textcolor{white}{a}\ \textcolor{white}{a}}}

\newcommand{\me}{\mathrm{e}}  %定义 对数常数e,虚数符号i,j以及微分算子d为直立体。
\newcommand{\mi}{\mathrm{i}}
\newcommand{\mj}{\mathrm{j}}
\newcommand{\dif}{\mathrm{d}}
\newcommand{\bs}{\boldsymbol}%数学黑体
\newcommand{\ds}{\displaystyle}
%通常我们使用的分数线是系统自己定义的分数线,即分数线的长度的预设值是分子或分母所占的最大宽度,如何让分数线的长度变长成,我们%可以在分子分母添加间隔来实现。如中文分式的命令可以定义为:
%\newcommand{\chfrac[2]}{\cfrac{\;#1\;}{\;#2\;}}
%\frac{1}{2} \qquad \chfrac{1}{2}

%选择题
\newcommand{\fourch}[4]{\\\begin{tabular}{*{4}{@{}p{3.5cm}}}(A)~#1 & (B)~#2 & (C)~#3 & (D)~#4\end{tabular}} % 四行
\newcommand{\twoch}[4]{\\\begin{tabular}{*{2}{@{}p{7cm}}}(A)~#1 & (B)~#2\end{tabular}\\\begin{tabular}{*{2}{@{}p{7cm}}}(C)~#3 &
		(D)~#4\end{tabular}}  %两行
\newcommand{\onech}[4]{\\(A)~#1 \\ (B)~#2 \\ (C)~#3 \\ (D)~#4}  % 一行

\renewcommand{\headrulewidth}{0pt}
\pagestyle{fancy}
\begin{document} % 在begin前面加了一个空格以免出现显示错误,编译时应该去掉
%\fancyhf{}
%\fancyfoot[CO,CE]{\vspace*{1mm}第\,\thepage\,页 , 共 ~\pageref{LastPage} 页}
%\sbox{\zdx}
%{\parbox{27cm}{\centering
%	座位号~\underline{\makebox[34mm][c]{}}~ 班~级\underline{\makebox[34mm][c]{}}~\CJKfamily{song} 学~号\underline{\makebox[44mm][c]{}}~\CJKfamily{song} 姓~名\underline{\makebox[34mm][c]{}} ~\\
%	\vspace{3mm}
%请在所附答题纸上空出密封位置。并填写试卷序号、班级、学号和 姓名\\
%%答题时学号
%\vspace{1mm}
%\dotfill{} 密\dotfill{}封\dotfill{}线\dotfill{} \\
%	}}
%	\reversemarginpar
	
\begin{spacing}{1.25}
	\begin{center}
\begin{LARGE}

~\underline{~2018 }\,年~\underline{小学数学}\,某大附中招生测试试卷\\

\end{LARGE}
\vspace{0.3cm}
(闭卷笔试\ \ 90 分钟)\\
	\vspace{0.5cm}
\begin{tabular}{|m{0.05\textwidth}|*{7}{m{0.05\textwidth}|}p{0.2\textwidth}|}
	\hline
\centering  题~号 & \centering 一 & \centering 二 & \centering 三 & \centering 四& \centering 五 & \centering 六 %& \centering 七 % & \centering 八 & \centering 九 & \centering 十
& \centering 总~分 & \makecell{阅卷教师} \rule{0pt}{3mm} \\
	\hline
	\centering 分~数 &  &  &  &  &  &  &  &  %&  &  &
	\rule{0pt}{8mm} \\\hline
				% \centering 计 &  &  &  &  &  &  &  &  &  &  & \\
				% \centering 分 &  &  &  &  &  &  &  &  &  &  & \\
				% \centering 人 &  &  &  &  &  &  &  &  &  &  & \\  \hline
\end{tabular}
\end{center}
\end{spacing}
\vspace{-0.5cm}
\setlength{\marginparsep}{1.7cm}
\putzdx %%装订线--奇页数

\section*{\hspace{5cm} 一、填空题~(每题~2 分, 共~ 20 分)}
\vspace{-1cm}

\begin{tabular}{|p{0.1\textwidth}|p{0.08\textwidth}|}
			\hline
			% after \\: \hline or \cline{col1-col2} \cline{col3-col4} ...
			\centering 阅卷人& \\
			\hline
			\centering 得~~分 &  \\
			\hline
		\end{tabular}
		\begin{enumerate}\setcounter{enumi}{0}
			\item $M*N~$表示$(M+N)~\div~2$,则(2017*2019)*2018= ~\underline{~~~~~~~~~~~~~}.
			
			\item 甲、乙两包糖的质量比是4:1,如果从甲包中取出13克放入乙包后,甲、乙两包糖的质量比是7:5,那么两包糖质量的总和是\underline{~~~~~~~~~~~~~}克.
			
			\item 将底面半径4分米,高3分米的圆柱形木料做成最大的圆锥,被切割掉的部分的体积是\underline{~~~~~~~~~~~~~}立方分米.
			
            \item 快、慢两车同时从甲、乙两地相对而行,经过6小时候在离终点40千米处两车相遇,相遇后两车仍以原速度行驶,快车又用4小时到达乙地,甲、乙两地的路程是\underline{~~~~~~~~~~~~~}千米.
			
			\item 一个正方体的表面积是$a~cm^2$,体积是$a~cm^3$,整个正方体的棱长是\underline{~~~~~~~~~~~~~}.

            \item 把浓度为95\%的究竟600毫升,稀释成浓度为75\%的酒精,需要加入\underline{~~~~~~~~~~~~~}毫升蒸馏水.

            \item 原计划从甲地到乙地每隔40米安装一根电线杆,加上两端共需61根;现在改成每隔60米安装一个电线杆,则需要购买\underline{~~~~~~~~~~~~~}根电线杆.

            \item 把$\dfrac{6}{7}$的分子减去3,要使分数的小小不变,分母应该减去\underline{~~~~~~~~~~~~~}.

            \item 在60.6千克药水中,药粉和谁的比是1:100,其中药粉有\underline{~~~~~~~~~~~~~}千克.

            \item 天平一端放着3块巧克力,另一端放着$\dfrac{1}{2}$块巧克力和60克的砝码,这是天平正好平衡,则一块巧克力重\underline{~~~~~~~~~~~~~}克.

		\end{enumerate}
\newpage

\vspace{1cm}
	\begin{spacing}{1.3}
		
		\section*{\hspace{5cm} 二、选择题~(每题~2 分,共~10 分)}
		\vspace{-1cm}
		\begin{tabular}{|p{0.1\textwidth}|p{0.08\textwidth}|}
			\hline
			% after \\: \hline or \cline{col1-col2} \cline{col3-col4} ...
			\centering 阅卷人& \\
			\hline
			\centering 得~~分 &  \\
			\hline
		\end{tabular}
		
\begin{enumerate}\setcounter{enumi}{10}
\vspace{0.3cm}

\item 一桶牛奶,喝了它的~$\dfrac{3}{5}$还多0.5升,这时还剩下3.5升,求这桶牛奶原有多少升?正确的列式是(~~~~).
\twoch{3.5$~\div~$(1~-~$\dfrac{3}{5}$)}{(3.5~+~0.5)$~\div~(1-\dfrac{3}{5}$)}{3.5$~\div(1-\dfrac{3}{5})~$-0.5}{3.5$~\div\dfrac{3}{5}~$}

\item 用3、4、5、6中任意两个数组成互质数,可组成(~~~~~).
\fourch{1对}{2对}{3对}{4对}

\item 足球门票50元一张,降价后观众增加2/3,收入增加1/6,一张门票降价(~~~~).
\fourch{12}{15}{14}{18}

\item 把一张长90cm,宽42cm的长方形铁板简称边长都是整厘米,面积都相等的小正方形铁片,恰好无剩余,至少要剪(~~~~)块.
\fourch{100}{105}{110}{108}

\item 已知x,y都是自然数,并且$\dfrac{x}{5}+\dfrac{y}{7}=\dfrac{43}{35}$, 那么x+y的值是(~~~~).
\fourch{6}{7}{8}{9}

			
\end{enumerate}

\section*{\hspace{4.5cm} 三、判断题:~正确$\surd$, 错误$\times$ (共~5分)}
\vspace{-1cm}
\begin{tabular}{|p{0.1\textwidth}|p{0.08\textwidth}|}
\hline
			% after \\: \hline or \cline{col1-col2} \cline{col3-col4} ...
\centering 阅卷人& \\
\hline
\centering 得~~分 &  \\
	\hline
\end{tabular}
\vspace{0.3cm}
\begin{enumerate}\setcounter{enumi}{15}
			
\item 某种手机的价格先降价5\%,又降价10\%,现价是原价的85\%.\hfill(~~~~~~~~)
\item 在含盐量为30\%的盐水先加7克水,再加3克盐,含盐量增大. \hfill(~~~~~~~~)
\item 周长相等的正方形和圆,正方形的面积比圆的面积小. \hfill(~~~~~~~~)
\item 把一个长方形的框架拉成一个平行四边形,他的面积不变,周长变小.\hfill(~~~~~~~~)
\item 一个小数精确到百分位是8.60,那么这个小数最大是8.599.\hfill(~~~~~~~~)		
\end{enumerate}

\newpage
		\putzdx %%装订线--奇页数
		
		\section*{\hspace{5cm} 四、计算题~(每题~5分, 共~20分)}
		\vspace{-1cm}
		\begin{tabular}{|p{0.1\textwidth}|p{0.08\textwidth}|}
			\hline
			% after \\: \hline or \cline{col1-col2} \cline{col3-col4} ...
			\centering  阅卷人&  \\
			\hline
			\centering 得~~分 &  \\
			\hline
		\end{tabular}
		\begin{enumerate}\setcounter{enumi}{20}
		
        \item $15\times\dfrac{1}{4}\times0.25+58\times25\%$.


		解:
        \vspace{2.5cm}

        \item  $2-[1-(\dfrac{3}{4}-\dfrac{3}{4}\div\dfrac{12}{5})]\div\dfrac{7}{8}$.


		解:
        \vspace{2.5cm}


        \item $\dfrac{2\times2}{1\times3}+\dfrac{4\times4}{3\times5}+\dfrac{6\times6}{5\times7}+\dfrac{8\times8}{7\times9}+\dfrac{10\times10}{9\times11}$.
			
        解:
        \vspace{2.5cm}

         \item $\dfrac{1}{6}+\dfrac{1}{24}+\dfrac{1}{60}+\dfrac{1}{120}+\dfrac{1}{210}$.
			
        解:
        \vspace{2.5cm}
\end{enumerate}
\newpage
\section*{\hspace{5cm} 五、面积计算~(每题~5 分, 共~ 10 分)}
\vspace{-1cm}
\begin{tabular}{|p{0.1\textwidth}|p{0.08\textwidth}|}
\hline
			% after \\: \hline or \cline{col1-col2} \cline{col3-col4} ...
\centering 阅卷人& \\
\hline
\centering 得~~分 &  \\
\hline
\end{tabular}
		
		\begin{enumerate}\setcounter{enumi}{24}
			\item 如下图所示,梯形面积是70平方厘米,下底是13厘米,求阴影部分的面积($\pi$取3).
			
			解:
            \vspace{2.5cm}
			
			\item 下图中$\Delta ABC$被线段$ED$分成甲、乙两部分,$AE=\dfrac{2}{5}AB,BD=\dfrac{1}{4}BC.$,请问:甲、乙两部分的面积比是多少?

           解:
            \vspace{2.5cm}
			

			
		\end{enumerate}
		\newpage
		\putzdx %%装订线--奇数页
		\section*{\hspace{5cm} 六、应用题~(每题~7 分,共~35分)}
		\vspace{-1cm}
		\begin{tabular}{|p{0.1\textwidth}|p{0.08\textwidth}|}
			\hline
			% after \\: \hline or \cline{col1-col2} \cline{col3-col4} ...
			\centering  阅卷人& \\
			\hline
			\centering 得~~分 &  \\
			\hline
		\end{tabular}

		\begin{enumerate}\setcounter{enumi}{26}

			\item 甲乙两个仓库共有粮食600吨,如果从甲仓库调出10\%,送入乙仓库后,甲、乙仓库的粮食质量比是3:2.求甲、乙两个仓库原来各有粮食多少吨?
			
           解:
            \vspace{3.5cm}

            \item 六年级三个班植树,任务分配是:甲班要植三个班植树总数的40\%,乙、丙两班植树棵数的比是5:2.当甲班植树200颗时,正好完成三个半植树总棵树的$\dfrac{2}{7}$.丙班应植树多少棵?

           解:
            \vspace{3.5cm}

            \item  甲乙两人分别从A、B两地同时出发,相向而行,出发时速度比是5:3,第一次相遇后,甲提速20\%,乙提速30\%,这样当甲到达B地时,乙离A地还有183千米,那么A、B两地之间的距离多少?
		\end{enumerate}
	

	\end{spacing}
\newpage

\begin{enumerate}\setcounter{enumi}{29}


			\item 甲有若干本书,乙借走了一半加3本,剩下的书,丙借走了1/3加2本,再剩下的书丁借走了1/2加三本,最后甲还有5本书,甲原来有多少本书?
			
           解:
            \vspace{3.5cm}

            \item 甲与乙班学生同时从学校出发去牧野公园,学校距公园57千米,甲班步行的速速是每小时7千米,乙班步行的速度是每小时9千米。学校有一辆汽车,它的速度是每小时63千米,这辆汽车恰好能坐一个班的学生,为了使两个班学生在最短时间内同时到达公园,那么甲班学生需要步行的距离是多少千米?

           解:
            \vspace{3.5cm}

            \item  甲、乙、丙三个人,甲每分钟行走120米,乙每分钟行走100米,丙每分钟行走70米,如果三人同时同向,从同地出发,沿周长是300米的圆形跑道行走,那几分钟之后,三个人又可以相聚?
		\end{enumerate}
	\clearpage
	
\end{document}
